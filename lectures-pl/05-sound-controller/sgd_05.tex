\documentclass{beamer}

\usepackage[polish]{babel}
\usepackage[utf8]{inputenc}
\usepackage[OT4]{fontenc}
\usepackage{algpseudocode}
\usepackage{beamerthemeshadow}
\usepackage{fancyvrb}
\usepackage{listings}
\usepackage{graphicx}
\usepackage{url}
\usepackage{shortcuts}

\beamertemplateballitem
\beamertemplatenumberedballsectiontoc

\hypersetup{%
  pdftitle={Symulacje i gry decyzyjne 5},%
  pdfauthor={Tadeusz Puźniakowski}}

\title[SGD 5]{Audio}

\author{Tadeusz Puźniakowski}

\institute{PJATK}
\date{2017}

\newcounter{minisection}
\newcommand{\minisectionframe}[1]{
\frame{
	\begin{center}
	#1
	\end{center}
}}

\newcommand{\smallhref}[1]{{\footnotesize\href{#1}{\url{#1}}}}

\begin{document}

\frame{\titlepage}


\begin{frame}[fragile]
	\BB{Muzyka w SDL -- wersja bez dodatków}
	Niskopoziomowo -- SDL rozpoczyna odtwarzanie i wywołuje callback który powinien uzupełnić bufor audio w momencie gdy potrzebne są kolejne próbki dźwięku.
	\EB
	\BB{Muzyka w SDL -- SDL\_mixer}
	Biblioteka pozwalająca na wygodniejsze korzystanie z podstawowego interfejsu audio SDL.
	\EB
\end{frame}


\end{document}

