\documentclass{beamer}

\usepackage[polish]{babel}
\usepackage[utf8]{inputenc}
\usepackage[OT4]{fontenc}
\usepackage{algpseudocode}
\usepackage{beamerthemeshadow}
\usepackage{fancyvrb}
\usepackage{listings}
\usepackage{graphicx}
\usepackage{url}
\usepackage{shortcuts}

\beamertemplateballitem
\beamertemplatenumberedballsectiontoc

\hypersetup{%
  pdftitle={Symulacje i gry decyzyjne 1},%
  pdfauthor={Tadeusz Puźniakowski}}

\title[SGD 1]{SDL i wstęp}

\author{Tadeusz Puźniakowski}

\institute{PJATK}
\date{2018}

\newcounter{minisection}
\newcommand{\minisectionframe}[1]{
\frame{
	\begin{center}
	#1
	\end{center}
}}

\newcommand{\smallhref}[1]{{\footnotesize\href{#1}{\url{#1}}}}

\begin{document}

\AtBeginSection[]
{
  \begin{frame}
    \frametitle{Spis treści}
    \tableofcontents[currentsection]
  \end{frame}
}


\frame{\titlepage}

\section{Co na części praktycznej}

\begin{frame}[fragile]
	\frametitle{Plan części praktycznej}
	\BB{Kontakt}
	tadeusz.puzniakowski@pjwstk.edu.pl
	\EB
\end{frame}

\begin{frame}[fragile]
	\frametitle{Plan części praktycznej}
	\BB{Co na zajęciach}
		Wykład jest przygotowaniem do lab. Zaliczenie tej części polega na napisaniu projektu (opowiem na ćwiczeniach).
		\begin{itemize}
		\item Elementy engine-u gry
		\item SDL
		\item Elementy fizyki w grach
		\item Kwestie związane z prezentacją gry (grafika, dźwięk, kontrolery)
		\item (jeśli wystarczy czasu) Elementy gry multiplayer
		\end{itemize}
	\EB
\end{frame}


\begin{frame}[fragile]
	\frametitle{Plan dzisiejszego wykładu}
	\BB{Co w planie na dzisiaj}
		\begin{itemize}
		\item Ogólnie o tym co to jest SDL
		\BI
		   \I Co to w ogóle jest?
		   \I Dlaczego nie DirectX
		   \I Kto korzysta z SDL i przykłady gier
		\EI
		\item Przykład "Witaj w Świecie"
		\item Wyświetlenie obrazka
		\item Obsługa klawiszy
		\item Przykład na żywo
		\end{itemize}
	\EB
\end{frame}
	
\section{Wstęp}

\begin{frame}[fragile]
	\frametitle{SDL}
	\BB{SDL}
	Simple DirectMedia Layer -- wieloplatformowa biblioteka programistyczna pozwalająca na niskopoziomową obsługę grafiki, dźwięku i kontrolerów gier.
	
	Polecam na początek: \url{https://icculus.org/SteamDevDays/}.
	\EB
	
	\BB{Dlaczego nie DirectX}
	Nie jest przenośny, oraz HelloWorld to koszmar:
	 \url{https://github.com/Microsoft/DirectX-Graphics-Samples/blob/master/Samples/Desktop/D3D12HelloWorld/src/HelloWindow/D3D12HelloWindow.cpp}
	\EB
\end{frame}

\begin{frame}[fragile]
	\frametitle{SDL}
	\BB{SDL -- kto z tego korzysta}
	Oto przykłady firm i wybranych gier korzystających z SDL
	\BI
	\I Valve -- Counter Strike Global Offensive
	\I THQ -- Painkiller
	\I Double Fine Productions -- Psychonauts
	\I Daedalic Entertainment -- Deponia
	\I (oraz oczywiście wiele innych)
	\EI
	\EB
\end{frame}


\begin{frame}[fragile]
	\frametitle{Trochę prostej teorii}
	\BB{Engine gry}
	To jest pojęcie raczej powszechnie znane - jest to oprogramowanie które prezentuje ,,świat gry''.
	W tym jest:
	\BI
	 \I Ładowanie świata gry
	 \I Generowanie obrazu 
	 \I Obsługa interfejsu użytkownika
	 \I Generowanie kolejnych zmian ,,świata gry'' -- fizyka (różniczki)
	\EI
	\EB
\end{frame}



\section{Programowanie dla wielu platform}

\begin{frame}[fragile]
	\BB{Elementy specyficzne dla danej platformy}
	Dobra praktyka
	\begin{verbatim}
	#if WINDOWS
	   cośtam dla windows
	#elif PLAYSTATION
	   cośtam dla playstation
	#elif LINUX
	   cośtam dla linux
	#else
	   #error Nieobsługiwana platforma
	#endif
	\end{verbatim}
	\EB
\end{frame}


\begin{frame}[fragile]
 	\BB{Witaj Świecie -- klasyk}
	\begin{verbatim}
    SDL_Init( SDL_INIT_EVERYTHING ); 
    auto window = SDL_CreateWindow( "Okienko SDL", 
             SDL_WINDOWPOS_UNDEFINED, 
             SDL_WINDOWPOS_UNDEFINED, 
             640, 480, 
             SDL_WINDOW_SHOWN ); 
    SDL_Delay(2000);
    SDL_DestroyWindow( window );
    SDL_Quit();
 	\end{verbatim}
 	\EB
\end{frame}

\begin{frame}[fragile]
 	\BB{Witaj Świecie -- wersja OpenGL}
	\begin{verbatim}
    SDL_Init( SDL_INIT_EVERYTHING ); 
    auto window = SDL_CreateWindow( "Okienko SDL", 
             SDL_WINDOWPOS_UNDEFINED, 
             SDL_WINDOWPOS_UNDEFINED, 
             640, 480, 
             SDL_WINDOW_SHOWN | SDL_WINDOW_OPENGL ); 
    SDL_GL_CreateContext(window);
    glClear(GL_COLOR_BUFFER_BIT);
    SDL_GL_SwapWindow(window);
    SDL_Delay(2000);
    SDL_DestroyWindow( window );
    SDL_Quit(); 
 	\end{verbatim}
 	\EB
\end{frame}


\begin{frame}[fragile]
	\BB{Inicjalizacja SDL}
	\begin{verbatim}
    SDL_Init( SDL_INIT_EVERYTHING ); 
 	\end{verbatim}
 	\EB
	\BB{Zwalnianie zasobów}
	\begin{verbatim}
    SDL_Quit(); 
 	\end{verbatim}
 	\EB
\end{frame}
\begin{frame}[fragile]
	\BB{Stworzenie okienka oraz kontekstu OpenGL}
	\begin{verbatim}
    auto window = SDL_CreateWindow( "Okienko SDL", 
             SDL_WINDOWPOS_UNDEFINED, 
             SDL_WINDOWPOS_UNDEFINED, 
             640, 480, 
             SDL_WINDOW_SHOWN | SDL_WINDOW_OPENGL ); 
    SDL_GL_CreateContext(window);
 	\end{verbatim}
 	SDL\_WINDOW\_OPENGL oznacza, że korzystamy z OpenGL
 	\EB
	\BB{Zniszczenie okienka}
	\begin{verbatim}
    SDL_DestroyWindow( window );
 	\end{verbatim}
 	\EB
\end{frame}
\begin{frame}[fragile]
	\BB{Wyświetlenie obrazu z backbufora OpenGL}
	\begin{verbatim}
    SDL_GL_SwapWindow(window);
 	\end{verbatim}
 	\EB
\end{frame}
\begin{frame}[fragile]
	\BB{Opóźnienie w wersji SDL}
	\begin{verbatim}
    SDL_Delay(2000);
 	\end{verbatim}
 	\EB
	\BB{Opóźnienie w wersji C++11}
	\begin{verbatim}
    sleep_for(milliseconds(33));
 	\end{verbatim}
 	Jeśli prefiksujemy wszystkie przestrzenie nazw:
	\begin{verbatim}
    std::this_thread::sleep_for ( 
         std::chrono::milliseconds(1000)
    );
 	\end{verbatim}
 	\EB
\end{frame}
\begin{frame}[fragile]
	\BB{Inicjalizacja renderera SDL}
	\begin{verbatim}
	SDL_Window *window;
	SDL_Renderer *renderer;
	SDL_CreateWindowAndRenderer(800, 600, 0, 
	       &window, &renderer);
 	\end{verbatim}
 	\EB
\end{frame}
\begin{frame}[fragile]
	\BB{Wyświetlenie piksela za pomocą renderera}
	\begin{verbatim}
	SDL_RenderDrawPoint(renderer, x, y);
	SDL_RenderPresent(renderer);
 	\end{verbatim}
 	\EB
\end{frame}
\begin{frame}[fragile]
	\BB{Pobranie stanu klawiatury}
	\begin{verbatim}
	const Uint8 *state = SDL_GetKeyboardState(NULL);
	if (state[SDL_SCANCODE_LEFT]) ....
 	\end{verbatim}
 	\EB
\end{frame}
\begin{frame}[fragile]
	\BB{Pętla zdarzeń (event loop)}
	\begin{verbatim}
	SDL_Event e;
	while( SDL_PollEvent( &e ) != 0 ) { 
	  if( e.type == SDL_QUIT ) { 
	    finishCondition = true;
	  } else if (e.type == SDL_KEYDOWN) {
	    if (e.key.keysym.sym == SDLK_ESCAPE)
	      finishCondition = true;
	  }
	}
 	\end{verbatim}
 	\EB
\end{frame}
\begin{frame}[fragile]
	\BB{Przykład na żywo}
	\IMG{9cm}{HOW-TO-DRAW-A-HORSE.jpg}
 	\EB
 	{\small{za \url{http://i.huffpost.com/gen/482429/HOW-TO-DRAW-A-HORSE.jpg}}}
\end{frame}


\end{document}

