\documentclass{beamer}

\usepackage[polish]{babel}
\usepackage[utf8]{inputenc}
\usepackage[OT4]{fontenc}
\usepackage{algpseudocode}
\usepackage{beamerthemeshadow}
\usepackage{fancyvrb}
\usepackage{listings}
\usepackage{graphicx}
\usepackage{url}
\usepackage{shortcuts}

\beamertemplateballitem
\beamertemplatenumberedballsectiontoc

\hypersetup{%
  pdftitle={Symulacje i gry decyzyjne 1},%
  pdfauthor={Tadeusz Puźniakowski}}

\title[SGD 4]{OpenGL}

\author{Tadeusz Puźniakowski}

\institute{PJATK}
\date{2017}

\newcounter{minisection}
\newcommand{\minisectionframe}[1]{
\frame{
	\begin{center}
	#1
	\end{center}
}}

\newcommand{\smallhref}[1]{{\footnotesize\href{#1}{\url{#1}}}}

\begin{document}

\frame{\titlepage}


\begin{frame}[fragile]
	\BB{OpenGL - plan wykładu}
		\BI
		\I Ustawienie parametrów sceny.
		\I Współrzędne.
		\I Wierzchołki.
		\I Wektory normalne.
		\I Współrzędne tekstury (mapa UV).
		\I Ładowanie tekstury.
		\I Indeksowanie wierzchołków.
		\I Shadery i VBO.
		\EI
	\EB
\end{frame}

\begin{frame}[fragile]
	\BB{Sieć}
		\BI
		\I Gniazdo sieciowe (socket).
		\I Gniazda strumieniowe (połączeniowe) i datagramowe.
		\I Gniazdo nasłuchujące.
		\I Blokująca i nieblokująca akceptacja połączenia.
		\I Port.
		\I Uzyskiwanie adresu na podstawie ciągu znaków.
		\I Podłączenie do zdalnego gniazda nasłuchującego.
		\I send, recv
		\I Protokół komunikacyjny.
		\EI
	\EB
\end{frame}


\end{document}

