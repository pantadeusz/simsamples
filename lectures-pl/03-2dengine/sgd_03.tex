\documentclass{beamer}

\usepackage[polish]{babel}
\usepackage[utf8]{inputenc}
\usepackage[OT4]{fontenc}
\usepackage{algpseudocode}
\usepackage{beamerthemeshadow}
\usepackage{fancyvrb}
\usepackage{listings}
\usepackage{graphicx}
\usepackage{url}
\usepackage{shortcuts}

\beamertemplateballitem
\beamertemplatenumberedballsectiontoc

\hypersetup{%
  pdftitle={Symulacje i gry decyzyjne 3},%
  pdfauthor={Tadeusz Puźniakowski}}

\title[SGD 3]{Grafika 2D}

\author{Tadeusz Puźniakowski}

\institute{PJATK}
\date{2016}

\newcounter{minisection}
\newcommand{\minisectionframe}[1]{
\frame{
	\begin{center}
	#1
	\end{center}
}}

\newcommand{\smallhref}[1]{{\footnotesize\href{#1}{\url{#1}}}}

\begin{document}

\AtBeginSection[]
{
  \begin{frame}
    \frametitle{Spis treści}
    \tableofcontents[currentsection]
  \end{frame}
}


\frame{\titlepage}

\section{Prosta symulacja}

\begin{frame}[fragile]
	\frametitle{Plan dzisiejszego wykładu}
	\BB{Prosty engine platformówki}
	\BI
	\I Podział kodu na moduły
	\I Uruchomienie renderera
	\I Ładowanie obrazków + cache
	\I Przeźroczystość
	\I Proste kolizje
	\I Elementy fizyki
	\I Wejście od użytkownika
	\I Obracanie obrazka
	\EI
	\EB
\end{frame}



\end{document}

