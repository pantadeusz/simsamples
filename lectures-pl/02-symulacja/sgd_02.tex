\documentclass{beamer}

\usepackage[polish]{babel}
\usepackage[utf8]{inputenc}
\usepackage[OT4]{fontenc}
\usepackage{algpseudocode}
\usepackage{beamerthemeshadow}
\usepackage{fancyvrb}
\usepackage{listings}
\usepackage{graphicx}
\usepackage{url}
\usepackage{shortcuts}

\beamertemplateballitem
\beamertemplatenumberedballsectiontoc

\hypersetup{%
  pdftitle={Symulacje i gry decyzyjne 2},%
  pdfauthor={Tadeusz Puźniakowski}}

\title[SGD 1]{Płynna animacja}

\author{Tadeusz Puźniakowski}

\institute{PJATK}
\date{2018}

\newcounter{minisection}
\newcommand{\minisectionframe}[1]{
\frame{
	\begin{center}
	#1
	\end{center}
}}

\newcommand{\smallhref}[1]{{\footnotesize\href{#1}{\url{#1}}}}

\begin{document}

\AtBeginSection[]
{
  \begin{frame}
    \frametitle{Spis treści}
    \tableofcontents[currentsection]
  \end{frame}
}


\frame{\titlepage}

\section{Prosta symulacja}

\begin{frame}[fragile]
	\frametitle{Plan dzisiejszego wykładu}
	\BB{Fizyka i proste efekty cząsteczkowe}
	\BI
	\I Pętla gry
	\I Elementy fizyki
	\I Dwa podstawowe podejścia do płynnej animacji (obliczanie czasu klatki, stała liczba klatek)
	\EI
	\EB
\end{frame}

\begin{frame}[fragile]
	\frametitle{Pętla gry}
	\BB{Game loop}
	Powtarza się aż do momentu, kiedy użytkownik zakończy grę
	\BI
	\I pętla zdarzeń (event loop)
	\I komunikacja z serwerem
	\I AI
	\I przemieszczenie postaci w grze
	\I rozwiązanie kwestii kolizjii
	\I wyświetlenie grafiki
	\I obsługa dźwięków
	\EI
	\EB
\end{frame}


\begin{frame}[fragile]
	\frametitle{Podstawowe wzory}
	\BB{Wzory do symulacji}
        $p = p + v * dt + \frac{a * dt^2}{2}$ \\
	$v = v + a * dt$ \\
	Sterujemy za pomocą przyśpieszenia, albo siły.
        Wyliczamy je w tym miejscu. Przypominam: $F=ma$
	\EB
	\BB{Opór powietrza}
	$F_{D}\,=\,{\tfrac {1}{2}}\,\rho \,v^{2}\,C_{D}\,A$, albo możemy na
nasze potrzeby sobie uprościć i stosować:
	$a = -Dv^2$.
	\EB
\end{frame}


\begin{frame}[fragile]
	\frametitle{Ciekawostki}
	\BB{A jak to bywa w grach?}
\BI	
\I \href{https://github.com/id-Software/Quake-III-Arena/blob/master/code/game/bg\_pmove.c}{bg\_pmove.c}, \href{https://github.com/raspberrypi/quake3/blob/master/code/client/cl\_main.c}{cl\_main.c}
\I \href{http://fabiensanglard.net/quake3/}{fabiensanglard.net --- quake3}
\EI
	\EB
\end{frame}


\begin{frame}[fragile]
	\frametitle{Sposoby radzienia sobie z czasem}
	\BB{Po prostu opóźnienie}
\begin{verbatim}
using namespace std::chrono;
duration<double> dt( 0.01 ); // w sekundach
...
std::this_thread::sleep_for( dt );
\end{verbatim}
	\EB
\end{frame}


\begin{frame}[fragile]
	\frametitle{Sposoby radzienia sobie z czasem}
	\BB{Stałe FPS}
\begin{verbatim}
using namespace std::chrono;
duration<double> dt( 0.015 ); // w sekundach
time_point<high_resolution_clock, duration<double> >
    prevTime = high_resolution_clock::now();
...
std::this_thread::sleep_until( prevTime + dt );
prevTime = prevTime + dt;
\end{verbatim}
	\EB
\end{frame}


\begin{frame}[fragile]
	\frametitle{Sposoby radzienia sobie z czasem}
	\BB{Pomiar czasu poprzedniej klatki - zmienne fps}
\begin{verbatim}
using namespace std::chrono;
time_point<high_resolution_clock, duration<double> > 
    prevTime = high_resolution_clock::now();
..
auto currentTime = high_resolution_clock::now();
dt = currentTime - prevTime;
prevTime = currentTime;
\end{verbatim}
	\EB
\end{frame}


\end{document}

