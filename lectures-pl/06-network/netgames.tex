\documentclass{beamer}

\usepackage[polish]{babel}
\usepackage[utf8]{inputenc}
\usepackage[OT4]{fontenc}
\usepackage{algpseudocode}
\usepackage{beamerthemeshadow}
\usepackage{fancyvrb}
\usepackage{listings}
\usepackage{graphicx}
\usepackage{url}
\usepackage{shortcuts}

\beamertemplateballitem
\beamertemplatenumberedballsectiontoc

\hypersetup{%
  pdftitle={Symulacje i gry decyzyjne - sieć},%
  pdfauthor={Tadeusz Puźniakowski}}

\title[Networking]{Gry sieciowe}

\author{Tadeusz Puźniakowski}

\institute{PJATK}
\date{\today}

\newcounter{minisection}
\newcommand{\minisectionframe}[1]{
\frame{
	\begin{center}
	#1
	\end{center}
}}

\newcommand{\smallhref}[1]{{\footnotesize\href{#1}{\url{#1}}}}

\begin{document}

\frame{\titlepage}


\begin{frame}[fragile]
	\BB{Sieć - plan wykładu}
		\BI
		\I Gniazda sieciowe - połączeniowe i bezpołączeniowe
		\I Architektury komunikacji
		\I Co gdzie kiedy - połączeniowe, bezpołączeniowe, 
		\I Ogólnie o rozwiązaniach w popularnych silnikach gier
		\EI
	\EB
\end{frame}


\begin{frame}[fragile]
	\BB{Gniazdo sieciowe}
	Dwukierunkowy punkt końcowy połączenia. Gniazdo reprezentowane jest najczęściej przez identyfikator gniazda (liczbę całkowitą).
	\EB
	\BB{Port sieciowy}
	Jest to liczba będąca parametrem gniazda sieciowego. Jest ona 16bitowa.
	\EB
\end{frame}

\begin{frame}[fragile]
	\BB{Gniazdo datagramowe}
	Gniazdo typu bezpołączeniowego. Prowadząc komunikację na takim gnieździe zawsze trzeba określić adres docelowy. Nie zabezpiecza transmisji przed utratą danych.
	\EB
	\BB{Gniazdo połączeniowe}
	Gniazdo które jest sparowane z gniazdem po drógiej stronie połączenia. Pozwala na kontrolę transmisji. Komunikacja odbywa się poprzez zapis i odczyt z gniazda.
	\EB
	\BB{Gniazdo surowe}
	Gniazdo które pozwala na ,,ręczne'' tworzenie pakietów. Najczęściej konieczne jest posiadanie uprawnień administratora aby z niego korzystać.
	\EB
\end{frame}

\begin{frame}[fragile]
	\frametitle{Modele komunikacji w grach}
	\BI
	\I Klient-serwer
	\I Peer-to-peer
	\EI
	
	\BI
	\I Symulacja po stronie serwera
	\I Symulacja po stronie klienta
	\EI
\end{frame}

\begin{frame}[fragile]
	\BB{Symulacja po stronie serwera -- Source Engine}
		\href{https://developer.valvesoftware.com/wiki/Source\_Multiplayer\_Networking}{link: Sieć w Source Engine}
		\BI
		\I Serwer trzyma i rozsyła stan gry
		\I Klienci wysyłają tylko wejście od użytkownika
		\I Klienci wyświetlają stan z przed chwili z serwera
		\I Klienci przewidują jaki bedzie ich własny stan, a korygują po otrzymaniu aktualnego od serwera
		\I Trzymanie 1 sekundy historii stanów gry przez serwer
		\EI
		Komunikacja po gniazdach UDP
	\EB
\end{frame}

\begin{frame}[fragile]
	\BB{Symulacja po stronie klienta -- Age of Empires}
	\href{https://www.gamasutra.com/view/feature/131503/1500_archers_on_a_288_network_.php?print=1}{link: Sieć w AOE}
		\BI
		\I Wszyscy klienci symulują to samo
		\I Stosunkowo długie tury ruchu
		\I Dostosowanie długości tury do możliwości sieci
		\I Lepiej jeśli jest większe opóźnienie niż zmienne
		\EI
		Komunikacja po gniazdach UDP
	\EB
\end{frame}

\begin{frame}[fragile]
	\BB{Komunikacja P2P -- Awesomenauts}
	\href{http://joostdevblog.blogspot.com/2014/09/core-network-structures-for-games.html}{link: Sieć w Awesomenauts}
		\BI
		\I Każda postać liczy samą siebię
		\I Wybrany klient symuluje inne elementy gry
		\EI
		Komunikacja po gniazdach UDP
	\EB
\end{frame}

\begin{frame}[fragile]
	\BB{Ogólna zasada}
		\BI
		\I Prawie zawsze korzystamy z UDP, zobacz \href{http://ithare.com/tcp-peculiarities-for-games-part-1/}{tcp-peculiarities-for-games}
		\I Liczby zmiennoprzecinkowe są ryzykowne, zobacz \href{https://blog.forrestthewoods.com/synchronous-rts-engines-and-a-tale-of-desyncs-9d8c3e48b2be}{problemy z desynchronizacją}
		\I Kompensacja opóźnień.
		\EI
	\EB
\end{frame}



% https://stackoverflow.com/questions/7286592/set-tcp-quickack-and-tcp-nodelay
% 

\end{document}

